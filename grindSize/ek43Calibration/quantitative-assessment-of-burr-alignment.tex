\documentclass[10pt,a4paper,twocolumn,notitlepage]{article}
\usepackage[latin1]{inputenc}
\usepackage{amsmath}
\usepackage{amsfonts}
\usepackage{amssymb}
\usepackage{titling}
\usepackage{lipsum}
\usepackage{graphicx}
\usepackage[left=2cm,right=2cm,top=2cm,bottom=2cm]{geometry}
\usepackage{hyperref}


\title{Quantitative Assessment Of A Mahlkonig EK-43 Burr Alignment}
\author{Tom Manion}

\begin{document}
\thispagestyle{empty}

\twocolumn[
  \begin{@twocolumnfalse}
    \maketitle
    
\textbf{Disclosure:}\\
\textit{This project would not be possible without the incredible program written by astrophysicist and specialty coffee brewing enthusiast, J. Gagne, at \url{https://coffeeadastra.com}. Much of his original source code, designed to understand the effects of particle size distribution on coffee taste, was used in the following experiment and can be found at: \url{https://github.com/jgagneastro/coffeegrindsize}. Supplementary code adapts data collected by his coffee grind size app, stored in .csv (comma separated values) files, into a readable form for multiple grind settings.}

    \begin{abstract}
      Recent variability in tasting notes during cupping of brewed coffee lead our team at Flight Coffee Co. Roasting Lab in Bedford, New Hampshire to suspect our Mahlkonig EK-43 Coffee Grinder needed an adjustment. This suspicion was verified using an open source program (discussed and linked in Disclosure). Subsequent burr adjustments made to the Mahlkonig EK-43 significantly improved the grinder's performance and our team's ability to achieve expected and consistent flavor profiles. A secondary portion of this study involved measurement of grind size at each EK-43 grind setting in an effort to compare these settings with settings of different grinders. The would help our team better support customers with different grinders who would like to improve extraction levels and flavor of their coffee. However, the EK-43's current configuration complicated this effort and new burrs will likely be needed to achieve precision in cross-product grind-size comparison. 
    \end{abstract}
  \end{@twocolumnfalse}
]




















\end{document}